\documentclass[10pt, letterpaper]{article}

\usepackage{
  amsfonts, % For \mathbb
  amsmath,
  lipsum
}

\input{tex/layout.tex}


%% -- Theorem Declarations ---------------------------------------------------------------------------------------------
%%

\usepackage{
  amsthm,
  thmtools,
  etoolbox
}

\input{tex/preamble/amsthm/amsthm-style.tex}

%% -- Theorem Declarations - Numbered ----------------------------------------------------------------------------------
%%

\xdeclaretheorem[
  name         = Definition,
  style        = definition-thmstyle,
  numberwithin = section,
]{definition}

\xdeclaretheorem[
  name         = Proposition,
  style        = proposition-thmstyle,
  numberwithin = section,
]{proposition}

\declaretheoremstyle[
  headfont      = \mdseries\bfseries,
  bodyfont      = \normalfont,
  spaceabove    = 1em,
  spacebelow    = 1.25em,
  postheadspace = 0.5em,
]{theorem-style}

\xdeclaretheorem[
  name         = Theorem,
  style        = theorem-style,
  numbered     = yes,
  numberwithin = section,
]{theorem}

\declaretheoremstyle[
  spaceabove    = 1em,
  spacebelow    = 1.25em,
  bodyfont      = \normalfont,
  postheadspace = 0.5em,
]{remark-style}

\xdeclaretheorem[
  name     = Remark,
  style    = remark-style,
  numbered = no,
]{remark}

%% -- Theorem Declarations - Unnumbered --------------------------------------------------------------------------------
%%

\xdeclaretheorem[
  name     = Definition,
  style    = definition-thmstyle,
  numbered = no,
]{definition*}

\xdeclaretheorem[
  name     = Proof,
  style    = proof-thmstyle,
  numbered = no,
]{proof*}

\xdeclaretheorem[
  name     = Proposition,
  style    = proposition-thmstyle,
  numbered = no,
]{proposition*}



\input{tex/preamble/fancyhdr.tex}
\input{tex/preamble/titlesec.tex}
\input{tex/preamble/xcolor.tex}

% Hyperref should always remain below all other imports.
\input{tex/preamble/hyperref.tex}

\begin{document}

%% -- Title page -------------------------------------------------------------------------------------------------------
%%

%% -- Title page -------------------------------------------------------------------------------------------------------
%%

\title{%
  Cantor Diagonalization Argument
  \\[1em]
  \normalsize{Compiled notes of the Cantor Diagonalization Argument}
}

\author{%
  \normalsize{Jacob Leach}
}

\date{%
  \normalsize
  \vspace{-1ex}Last updated: \today
}

\maketitle
\label{sec:contents}
\tableofcontents

\pagebreak

%% -- Sections ---------------------------------------------------------------------------------------------------------
%%

\section{Countable and Uncountable Sets}

\subsection{Countable Sets}

\begin{definition}[Countable Set]
A set $X$ is \textit{countable} if any of the following statements is true:

\begin{itemize}
  \item For a subset of the natural numbers $Y \subseteq \mathbb{N}$, there exists a bijection $f : X \to Y$.
  \item There exists a surjective function $f : X \to \mathbb{N}$.
  \item There exists an injective function $f : \mathbb{N} \to X$.
\end{itemize}
\end{definition}

\begin{definition}[Countably Finite Set]
A set $X$ is \textit{countably finite} if for a natural number $n \in \mathbb{N}$ and a set $Y = \{1, 2, ..., n\}$ there
exists mappings $f : X \to Y$ and $g : Y \to X$ such that $g \circ f = \text{id}_X$. That is to say, $X$ is
\textit{countably infinite} if there is a bijection between $X$ and a sequential subset of $\mathbb{N}$.
\end{definition}

\subsection{Properties of Countable Sets}

\begin{theorem}
For a countable set $X$ and a subset $Y$ of $X$, $Y$ is countable.

\textbf{Proof.} The set $X$ is countable, hence there exists a mapping $f: \mathbb{N} \to X$ and
$f^{-1} : X \to \mathbb{N}$ such that $f^{-1} \circ f = id_X$ - that is, $f$ maps from $\mathbb{N}$ to $X$ one-to-one.
Assume that $Y$ is uncountable and therefore, there is no bijection between $Y$ and $\mathbb{N}$. Since $Y \subseteq X$,
it follows that either $Y = X$ or $Y \subset X$. In the case that $Y = X$, we have a contradiction since $Y$ is assumed
to be uncountable while that $X$ is known to be countable and $Y = X$. In the case that $Y \subset X$, we have a have
a contradiction since $Y$ is assumed to be uncountable while the mapping $f|_Y : Y \to \mathbb{N}$ is bijective since
$f^{-1} \circ f|_Y = id_Y$. Therefore, $Y$ is countable. \qed

\end{theorem}

\subsection{Uncountable Sets}

\begin{definition}[Countably Infinite Set]
A set $X$ is \textit{countably infinite} if there exists mappings $f : X \to \mathbb{N}$ and $g : \mathbb{N} \to X$ such
that $g \circ f = \text{id}_X$. That is to say, $X$ is countable and there exists a bijection between $X$ and
$\mathbb{N}$.
\end{definition}

\begin{definition}[Uncountable Set]
A set $X$ is \textit{uncountable} if there is no injective mapping from $X$ to $\mathbb{N}$, hence there is no
bijection between $X$ and $\mathbb{N}$. An \textit{uncountable set} is said to be \textit{uncountably infinite}.
\end{definition}

\begin{remark}
There is no set that is \textit{uncountable} and \textit{finite}. This is a consequence of the definitions for an
\textit{uncountable set} and a \text{countably finite set} being mutually exclusive around the existence of a bijection
to $\mathbb{N}$.
\end{remark}

\subsection{Properties of Uncountable Sets}

\begin{theorem}
If $X$ is uncountable and $X \subseteq Y$, then $Y$ is uncountable.

\textbf{Proof.} Assume that $Y$ is countable, hence there exists mappings $f : Y \to \mathbb{N}$ and
$g : \mathbb{N} \to Y$ such that $g \circ f = \text{id}_Y$. Since $X \subseteq Y$, we have that either the case that
$X = Y$ or $X \subset Y$. In the case that $X = Y$ we have a contradiction by the fact that $X$ is known to be
uncountable and $Y$ is assumed to be countable. Otherwise, we have that $X \subset Y$ and a mapping
$f|_X : X \to \mathbb{N}$ that is bijective since $g \circ f|_X = id_X$. This contradicts the assumption that $X$ is
uncountable. Therefore, $Y$ is uncountable. \qed
\end{theorem}

\section{Existence of Uncountable Sets}

Cantor illustrated his diagonalization argument in the paper "Uber ein elementare Frage der Mannigfaltigkeitslehre"
which was published to the journal of the "Deutsche Mathematiker-Vereinigung" (German Mathematical Union) in 1891. The
German Mathematical Union was an organization that was founded by Cantor in 1890. Cantor notes in "Uber ein elementare
Frage der Mannigfaltigkeitslehre" that the diagonalization argument laid out in the paper was not the first to provide a
proof of the existence uncountably infinite sets, but it was the first proof to do so without considering the
irrationals. An earlier paper by Cantor "On a property of a set [Inbegriff] of all real algebraic numbers" published in
1877 was the first proof that there are sets that are uncountably infinite and further that the set of real numbers is
uncountably infinite.

\subsection{Diagonalization Argument}

\begin{theorem}
There exists a set that is uncountably infinite.

\textbf{Proof.} Let $M$ be the set containing two elements, that is $M = \{0, 1\}$. Let $M^*$ be the set of all infinite
strings $x_1 x_2 x_3 \ldots$ with atoms $x_i \in M$. Assume that all sets - including $M^*$ - are countable. Hence the
enumeration:

\begin{align*}
M^*_1 &= 00000 \ldots 0 \dots \\
M^*_2 &= 11111 \ldots 1 \dots \\
M^*_3 &= 10101 \ldots 0 \dots \\
M^*_4 &= 01010 \ldots 1 \dots \\
M^*_5 &= 11011 \ldots 0 \dots \\
\vdots
\end{align*}

Should enumerate every elements of the set $M^*$ exactly once. To show that $M^*$ is not countable, we show that there
exists a string $M^*_0$ that cannot be equal to any string $M^*_n$ for $n \in 1, 2, 3, ...$. Therefore proving that
there is no bijective mapping from $\mathbb{N}$ to $M^*$ by contradiction; and therefore, proving that $M^*$ is
uncountable. For this, let there be an enumeration

\begin{align*}
M^*_1 &= x_{11} x_{12} x_{13} x_{14} x_{15} \ldots x_{1n} \dots \\
M^*_2 &= x_{21} x_{22} x_{13} x_{24} x_{25} \ldots x_{2n} \dots \\
M^*_3 &= x_{31} x_{32} x_{33} x_{34} x_{35} \ldots x_{3n} \dots \\
M^*_4 &= x_{41} x_{42} x_{43} x_{44} x_{45} \ldots x_{4n} \dots \\
M^*_5 &= x_{51} x_{52} x_{53} x_{54} x_{55} \ldots x_{5n} \dots \\
\vdots
\end{align*}

where $x_{ij}$ denotes an atom, that is $x_{ij} \in M$ for $i, j \in \mathbb{N}$. Furthermore, define the unary
operation $\neg : M \to M$ be the compliment of an atom:

\begin{align*}
\neg 0 &= 1 \\
\neg 1 &= 0
\end{align*}

Now define the string $M^*_0 = x_{01} x_{02} x_{03} \dots x_{0n} \dots$ such that $x_{0n} = \neg x_{nn}$. Given this we
can see that for all $n \in {1, 2, 3, \dots}$, there is no $M^*_n$ satisfying the equation $M^*_0 = M^*_n$. If there
were some $M^*_n$ satisfying $M^*_0 = M^*_n$, then $x_{0n} = x_{nn}$ for all $n \in \mathbb{N}$, which is a
contradiction by the definition of $M^*_0$. Therefore, $M^*$ is uncountable. \qed
\end{theorem}

\pagebreak

\begin{definition}[Algebraic Real Number]
An \textit{algebraic real number} is the root of a non-zero polynomial with coefficients in $\mathbb{Q}$ and with
finite degree of at least $1$. That is, algebraic real number a root to a polynomial of the form

\begin{equation*}
a_n x^n + a_{n-1} x^{n-1} + \ldots + a_1 x + a_0 = 0 \text{,}
\end{equation*}

for all $1, \dots, n$ and $n \in \mathbb{N}$.
\end{definition}

\begin{theorem}
The set of algebraic real numbers is uncountable.

\textbf{Proof.} Let $D$ be the set of all "digits", that is defined as $D = {0, 1, 2, 3, \dots, 9}$. Moreover, let
$D^*$ be the set of all decimal expansions, that is defined $D^* = d_{1} d_{2} d_{3} \ldots$ and $d_{n} \in D$. Let the
function $suc : D \to D$ be the "digit successor" function which is defined as:

\begin{align*}
suc(0) &= 1 \\
suc(1) &= 2 \\
suc(2) &= 3 \\
&\vdots \\
suc(8) &= 9 \\
suc(9) &= 0
\end{align*}

Now, let $f : D \to R$ be an injective function that maps a digit to the corresponding real number, that is:

\begin{align*}
f(0) &= 0 \\
f(1) &= 1 \\
f(2) &= 2 \\
&\vdots \\
f(9) &= 9
\end{align*}

Lastly, define a mapping $\text{dec}$ from $D^*$ to the set of algebraic real numbers as:

\begin{equation*}
\text{dec}(d_{1} d_{2} d_{3} \ldots) = \sum_{n=1}^{\infty} \frac{1}{10^{n}} f(d_{n})
\end{equation*}

Given this, we can see that the decimal expansion $999 \ldots$ under $\text{dec}$ corresponds to the
algebraic real number $1$

\begin{equation*}
\text{dec}(999 \ldots) &= \sum_{n=1}^{\infty} \frac{1}{10^{n}} f(9) \\
= \sum_{n=1}^{\infty} \frac{1}{10^{n}} 9 \\
= 0.999 \ldots
= 1
\end{equation*}

Now, consider the following enumeration of decimal expansions:

\begin{align*}
D^*_1 &= 00000 \ldots \\
D^*_2 &= 12345 \ldots \\
D^*_3 &= 23456 \ldots \\
D^*_4 &= 34567 \ldots \\
D^*_5 &= 45678 \ldots \\
\vdots
\end{align*}

Proof that the set of algebraic real numbers is uncountable can be shown similarly to as is done in theorem $2.1$: by
assuming that the set of algebraic real numbers is countable and then showing that for any enumeration of decimal
expansions $D^*_n$ that is one-to-one with $\mathbb{N}$ - that is, there is $D^*_n$ for all $n \in \mathbb{N}$ - there
exists a decimal expansion $D^*_0$ that is not equal to any $D^*_n$ by construction. Thereby proving that the set of
algebraic real numbers is uncountable by contradiction. To carry this out, define $D^*_0 = d_{01} d_{02} d_{03} \ldots$
such that $d_{0n} = suc(d_{nn})$. Given this, we can see that for all $n \in {1, 2, 3, \dots}$, there is no $D^*_n$
satisfying the equation $D^*_0 = D^*_n$. If there were some $D^*_n$ satisfying $D^*_0 = D^*_n$, then $d_{0n} = d_{nn}$
for all $n \in \mathbb{N}$, which is a contradiction by the definition of $D^*_0$. Therefore, the set of algebraic real
numbers is uncountable. \qed
\end{theorem}

\begin{theorem}
The set of real numbers is uncountable.

\textbf{Proof.} The set of real numbers is the union of the set of algebraic real numbers and the set of transcendental,
hence the set of algebraic real numbers is a subset of $\mathbb{R}$. Thus it follows from theorem $2.2$ and theorem $2.1$
that $\mathbb{R}$ must also be uncountable. \qed
\end{theorem}

\end{document}